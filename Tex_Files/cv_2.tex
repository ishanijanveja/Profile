%%%%%%%%%%%%%%%%%%%%%%%%%%%%%%%%%%%%%%%%%
% Medium Length Graduate/Under-graduate Curriculum Vitae
% LaTeX Template
% Version 1.1 (9/12/12)
%
% This template has been downloaded from:
% http://www.LaTeXTemplates.com
%
% Original author:
% Rensselaer Polytechnic Institute (http://www.rpi.edu/dept/arc/training/latex/resumes/)
%
% Important note:
% This template requires the res.cls file to be in the same directory as the
% .tex file. The res.cls file provides the resume style used for structuring the
% document.
%
%%%%%%%%%%%%%%%%%%%%%%%%%%%%%%%%%%%%%%%%%

%----------------------------------------------------------------------------------------
%	PACKAGES AND OTHER DOCUMENT CONFIGURATIONS
%----------------------------------------------------------------------------------------

\documentclass[margin, 10pt]{res} % Use the res.cls style, the font size can be changed to 11pt or 12pt here

\usepackage{helvet} % Default font is the helvetica postscript font
%\usepackage{newcent} % To change the default font to the new century schoolbook postscript font uncomment this line and comment the one above

\setlength{\textwidth}{5.1in} % Text width of the document

\begin{document}

%----------------------------------------------------------------------------------------
%	NAME AND ADDRESS SECTION
%----------------------------------------------------------------------------------------

\moveleft.5\hoffset\centerline{\large\bf Ishani Janveja} % Your name at the top
 
\moveleft\hoffset\vbox{\hrule width\resumewidth height 1pt}\smallskip % Horizontal line after name; adjust line thickness by changing the '1pt'
 
\moveleft.5\hoffset\centerline{A-904, Gokul Apartment, Plot-5B, Sector 11, Dwarka} % Your address
\moveleft.5\hoffset\centerline{New Delhi, India - 110075}
\moveleft.5\hoffset\centerline{Phone No. : +91 9871297816}
\moveleft.5\hoffset\centerline{Email : ishani.janveja@gmail.com}
%----------------------------------------------------------------------------------------

\begin{resume}

%----------------------------------------------------------------------------------------
%	Interests
%----------------------------------------------------------------------------------------
 
\section{INTERESTS}  

Deep Learning, Computer Vision, Artificial Intelligence, Image/Video Processing, Product Designing

%----------------------------------------------------------------------------------------
%	Education 
%----------------------------------------------------------------------------------------
\section{EDUCATION} 
{\sl \bf{Bharati Vidyapeeth's College of Engineering}} \hfill 2015-2019(anticipated)\\ Bachelor of Technology in Electronics and Communication\\
Current CGPA Score - 8.9/10.0

{\sl \bf{Venkateshwar International School, Delhi}}
\hfill 2013-2015 \\Higher Secondary Education\\ 91.8\% in AISSCE 2015

{\sl \bf{Kalpa Co-Educational School, Hyderabad}}\\High School(Class 10)\\School Topper with 94.5\% in ICSE Board Examination 2013

%----------------------------------------------------------------------------------------
%	Training and Internships 
%----------------------------------------------------------------------------------------

\section{TRAINING \& \\ INTERNSHIPS}
{\sl \bf{Trainee at IIT Delhi}}\hfill Jan. 2018\\
Winter course on Deep Learning

{\sl \bf{Intern at IIT Delhi for Celestini Project India}}\hfill Jun. 2017-Jul. 2017\\
Led by : Dr. Aakanksha Chowdhery(Princeton University) and Dr. Brejesh Lall(IIT Delhi). Sponsored by Marconi Society, Google and IIT Delhi. 

{\sl \bf{Trainee at Cyborg Labs, Delhi}}\hfill Jul. 2016-Jun. 2016\\
Comprehensive course on Embedded Systems

%----------------------------------------------------------------------------------------
%	Technical Skills 
%----------------------------------------------------------------------------------------

\section{TECHNICAL \\ SKILLS} 

{\sl Programming Languages:} Python, C, Embedded C, MATLAB, LaTex. \\
{\sl Hardware Descriptive Language:} VHDL. \\
{\sl Embedded Platforms:} Raspberry Pi, Arduino, Firebird V, TI MSP 430. \\
{\sl Softwares and Libraries:} OpenCV, TensorFlow, Blender (GUI and Scripting), Mentor Graphics QuestaSim, OrCAD Capture/PSpice. 


%----------------------------------------------------------------------------------------
%	Awards 
%----------------------------------------------------------------------------------------

\section{AWARDS \& HONORS}
{\sl Paul Baran Young Scholars \bf{Celestini Prize India}} \hfill Nov. 2017 \\Awarded to honor the demonstration of innovation and blue-sky thinking in developing a telecom-based solution to a socioeconomic challenge. 

{\sl {\bf Certificate of Merit} at {\bf e-Yantra Robotics Competition (eYRC)}} \hfill Mar. 2017\\ For standing {\bf 4th out of 160 national teams} in theme Bothoven.

{\sl {\bf Scholar's prize} for standing {\bf 1st at school level} in ICSE Board Examination \hfill 2013}

%----------------------------------------------------------------------------------------
%	Publications and Projects 
%----------------------------------------------------------------------------------------
\newpage
\section{PUBLICATIONS}
[1] A. Arora, {\bf I. Janveja} and Brejesh Lall. "labelVDOS: label Very Dense Object Sequences." \textit{submitted to Computer Vision and Pattern Recognition Workshops (CVPRW) 2018}.

[2] N. Garg, {\bf I. Janveja}, D. Malhotra, C. Chawla, P. Gupta, H. Bansal, A. Chowdhery, P. Mukherjee and Brejesh
Lall, "DRIZY- Collaborative Driver Assistance Over Wireless Networks," \textit{submitted to ACM/IEEE International Conference on Internet-of-Things Design and Implementation 2018}.

[3] {\bf I. Janveja}, N. Garg, C. Chawla and J. Parikh. "AQUACOM : Underwater Visible Light Communication". \textit{Computing for Nation Development (INDIACom 2018), India}.

[4] N. Garg, {\bf I. Janveja}, D. Malhotra, C. Chawla, P. Gupta, H. Bansal, A. Chowdhery, P. Mukherjee, and Brejesh Lall. "Poster: DRIZY: Collaborative Driver Assistance Over Wireless Networks." \textit{Proceedings of the 23rd Annual International Conference on Mobile Computing and Networking. ACM, 2017}.

% \vspace{-2.5 mm}

% A vehicle-pedestrian and vehicle-vehicle collision alert system prototyped on Raspberry Pi alongside a smartphone application to infer speed and position data of vehicles as well as provide alerts.

\section{PROJECTS}
- {\bf IITD Winter Project 2017 - Depth Mapping of 2D Images using CNNs}\\
Transfer learning by using a VGG16 model pretrained on ImageNet dataset to generate depth map from a single monocular image. Prototyped using Keras.   

% - {\bf DRIZY : A Collaborative Driver Assistance System over Wireless Networks}\\
% A vehicle-pedestrian and vehicle-vehicle collision alert system prototyped on Raspberry Pi alongside a smartphone application to infer speed and position data of vehicles.

- {\bf Model a Terrain (eYRC 2016) : Planet Terrain Analysing and Modelling} \\
Generation of a map of the arena traversed by an Explorer Bot on a remote system. Prototyped on Firebird V 

- {\bf Bothoven (eYRC 2016) : Inter-Robot Communication for Cooperative Task Management} \\ Wireless communication using ZigBee technology to establish the collaborative task of striking a sequence of notes placed on an arena. 

- {\bf Visible Light Communication}\\
Transfer textual data to a remote receiver using a laser diode transmitter. Prototyped using Arduino Mega 2560. 
%- {\bf Gaming Arcade using cost effective Pressure Sensors for feedback

%----------------------------------------------------------------------------------------
%	Leadership Experiences 
%----------------------------------------------------------------------------------------

\section{LEADERSHIP EXPERIENCE}
{\sl {\bf Vice Chairperson}}\hfill Aug. 2017-present \\
Robotics and Automation Society, BVP-IEEE Student Branch, Delhi.

{\sl {\bf Student Representative}} \hfill Aug. 2016-2017\\ BVP-IEEE Student Branch, Delhi.

{\sl {\bf School Prefect}}\hfill 2014-2015\\
Venkateshwar International School, Delhi.

{\sl {\bf Head Girl}} \hfill 2012-2013\\ Kalpa Co-Educational School, Hyderabad.

%----------------------------------------------------------------------------------------
%	Hobbies
%----------------------------------------------------------------------------------------

\section{HOBBIES}
I seek pleasure in creating (just about anything). The idea of sharing stories through just pictures amuses me, hence my love for photography. I relish solitude as much as I like talking to people and most of my conversations generally begin with or include music apart from talks about science, upcoming trends in technology or interesting things that I come across on the Internet. I enjoy swimming as a sport and sharing my philosophies with the few like-minded people that I have around me.

\end{resume}
\end{document}
